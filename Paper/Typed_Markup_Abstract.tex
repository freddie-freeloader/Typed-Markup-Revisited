\begin{abstract}
  This paper is mostly a case study of the tagless-final style—which is a
  solution to the \emph{Expression Problem}. We describe how the tagless-final
  encoding can help to create truly extensible representations for markup
  documents. This is not a new finding and was first used in the Haskell project
  \emph{HSXML}.

  We provide a comparison of the tagless-final encoding with the \emph{algebraic
    data type} encoding—that \emph{Pandoc} is using—and describe the essential
  implementation techniques that HSXML’s implementation is based on to create a
  context aware encoding. This \emph{context aware tagless-final encoding} has
  great potential for creating a representation, that is
  \begin{itemize}
  \item truly extensible—i.e. in the dimension of \emph{constructors} and the
    dimension of \emph{observations}.
  \item provides guarantees in regards to the well-formedness of the created
    abstract syntax. These guarantees is ensured by the type system of the host
    language.
  \end{itemize}
\end{abstract}
